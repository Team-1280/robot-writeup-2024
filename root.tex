\documentclass[12pt, letterpaper]{article}
\usepackage{tabularx} % extra features for tabular environment
\usepackage{amsmath}  % improve math presentation
\usepackage{graphicx} % takes care of graphic including machinery
\usepackage[margin=1in,letterpaper]{geometry} % decreases margins
\usepackage{cite} % takes care of citations
\usepackage[final]{hyperref} % adds hyper links inside the generated pdf file
\hypersetup{
  colorlinks=true,       % false: boxed links; true: colored links
  linkcolor=blue,        % color of internal links
  citecolor=blue,        % color of links to bibliography
  filecolor=magenta,     % color of file links
  urlcolor=blue
}
\usepackage{blindtext}
%++++++++++++++++++++++++++++++++++++++++


\begin{document}

% working title, update later
\title{Building Robot 2024}
\author{Youwen Wu, Warren Lin}
\date{\today}
\maketitle

\begin{abstract}
  Team 1280 designed an innovative robot for the 2024 FRC Crescendo
competition through an interdisciplinary and iterative process. Key
features include a precision ball intake, high-speed launcher, agile
drivetrain, and advanced autonomous capabilities. This abstract details
the team's engineering approach, from initial brainstorming to final
validation, that resulted in a high-performing robot excelling in the
dynamic competition environment.
\end{abstract}


\section{Introduction}

Our 2024 robot incorporates numerous 

\section{Background}

Give a brief summary of the physical theory, include any equations necessary, and cite any references you want to include. Here is how you insert an equation. According to
references the dependence of interest is given
by
\begin{gather*}
  \mathcal{L} =  \frac{1}{2} m \ell^2 {( \dot{\theta}+\dot{\phi}_0)}^2 - m g_e(t) \ell \cos(\theta)\\
  \\
  m\ell^2 (\ddot{\theta} + \ddot{\phi}_0) = mg_e\ell\sin(\theta)\\
\end{gather*}
\begin{equation}
  \ddot{\phi}(t) = -\frac{g_e(t)}{\ell} \sin\left(\phi(t)-\phi_0(t)\right)
  \label{Eq:equation1} %the label lets you refer to the equation later
\end{equation}

\blindtext{} %delete this line

\section{Methods}


Give a schematic of the experimental setup{(s)} used in the experiment (see
figure). Give the description of  abbreviations
either in the figure caption or in the text. Write a description of what is
going on.

%\begin{figure}[ht] 
% read manual to see what [ht] means and for other possible options
%        \centering \includegraphics[width=0.8\columnwidth]{sr_setup}
% note that in above figure file name, "sr_setup",
% the file extension is missing. LaTeX is smart enough to find
% apropriate one (i.e. pdf, png, etc.)
% You can add this extention yourself as it seen below
% both notations are correct but above has more flexibility
%\includegraphics[width=1.0\columnwidth]{sr_setup.pdf}
%        \caption{
%                \label{fig:samplesetup} % spaces are big no-no withing labels
% things like fig: are optional in the label but it helps
% to orient yourself when you have multiple figures,
% equations and tables
%                Every figure MUST have a caption.
%        }
%\end{figure}

and eventually arrived to the
balanced photodiode as seen in the figure.


\section{Results}

In this section you will need to show your experimental results. Use tables and
graphs when it is possible.


Analysis of equatio shows \ldots

\blindtext{}

For example, it is easy to conclude that the
experiment and theory match each other rather well if you look at


\section{Conclusions}
Here you briefly summarize your findings. Did you learn any new physics? Was everything as expected?

\blindtext{}

\section{Future Work}
Since you had limited time to work on this project, what questions are left outstanding? What would be your next steps?

\blindtext{}

%++++++++++++++++++++++++++++++++++++++++
% References section will be created automatically 
% with inclusion of "thebibliography" environment
% as it shown below. See text starting with line
% \begin{thebibliography}{99}
% Note: with this approach it is YOUR responsibility to put them in order
% of appearance.

\begin{thebibliography}{99}

  \bibitem{melissinos}
  A.~C. Melissinos and J. Napolitano, \textit{Experiments in Modern Physics},
  (Academic Press, New York, 2003).

  \bibitem{Cyr}
  N.\ Cyr, M.\ T$\hat{e}$tu, and M.\ Breton,
  % "All-optical microwave frequency standard: a proposal,"
  IEEE Trans.\ Instrum.\ Meas.\ \textbf{42}, 640 (1993).

  \bibitem{Wiki} \emph{Expected value},  available at
  \texttt{http://en.wikipedia.org/wiki/Expected\_value}.

\end{thebibliography}


\end{document}