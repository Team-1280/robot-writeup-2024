% chktex-file 1
\documentclass[12pt, letterpaper, two]{article}
\usepackage{tabularx} % extra features for tabular environment
\usepackage{amsmath}  % improve math presentation
\usepackage{graphicx} % takes care of graphic including machinery
\usepackage[margin=1in,letterpaper]{geometry} % decreases margins
\usepackage[final]{hyperref} % adds hyper links inside the generated pdf file
\usepackage[sorting=none]{biblatex}
\hypersetup{
  colorlinks=true,       % false: boxed links; true: colored links
  linkcolor=blue,        % color of internal links
  citecolor=blue,        % color of links to bibliography
  filecolor=magenta,     % color of file links
  urlcolor=blue
}
\usepackage{blindtext}

%++++++++++++++++++++++++++++++++++++++++
\addbibresource{citations.bib}

\begin{document}

% working title, update later
\title{Building Robot 2024 (working title)}
\author{Youwen Wu, Warren Lin}
\date{\today}
\maketitle

\begin{abstract}
  Team 1280 designed an innovative robot for the 2024 FRC Crescendo
  competition through an interdisciplinary and iterative process. Key
  features include a precision ring intake, high-speed launcher, agile
  swerve drivebase, and advanced autonomous capabilities. This writeup
  details the team's engineering approach, from initial brainstorming to
  final validation, that resulted in a high-performing robot excelling
  in the competition environment.
\end{abstract}


\section{Introduction}

Our 2024 robot incorporates numerous new technologies we've been
exploring this year. Breaking a longstanding tradition of using a
rudimentary tank drive system, we used a 4-wheel swerve drivebase that
enables much more nimble and precise robot movements. We also were able
to implement computer vision, upgrading from a Microsoft webcam to two
limelights with a Google Coral TPU to identify game pieces, april tags,
and other visual fiducials. To create all of these features, we
underwent a massive overhaul of our prototyping and build process. In
this writeup, we'll discuss all of the work we accomplished this season
and our process.

\section{Innovation}
It's useful for us to first briefly present some of our ``greatest
hits'' in robot engineering and innovation this season, and then go into
further depth on our specific implementation in later sections.

\subsection{Drivebase}

Switching from tank drive to swerve drive was a big decision, and meant
we had to do a lot of things differently. First, we couldn't draw on any
previous designs or code, and basically had to start fresh with a brand
new drivebase design. Luckily, our Rev Robotics MAXSwerve
modules~\cite{revswervemodule} had ample documentation and integration
with WPILib, so we were able to build our drivebase prior to kickoff,
and begin writing a library for controlling it early into the season.

% TODO: add an image of the swerve drivebase


\subsection{Vision}

We vastly expanded our computer vision capabilities this year. In
2023, we had a Microsoft webcam which streamed a low-resolution camera
feed to the driver dashboard, and had no practical application for even
basic vision tasks, such as identifying visual fiducials like april
tags. This year, we got our hands on two Limelight v3s~\cite{limelight},
which are a full low/no code computer vision solution, with
out-of-the-box april tag detection. It also integrates seamlessly with
our new Coral TPU~\cite{coral}, which provides hardware acceleration for
Limelight to run neural networks such as note classifiers, which enable
the detection of game pieces via machine learning algorithms, previously
unfeasible for us. We also experimented with Microsoft's Kinect system,
which, though originally designed for motion gaming on the Xbox, has
proven to be an extremely powerful depth sensor and motion capture tool,
a reputation proven by its use in research laboratories around the
world~\cite{kinectpaper}. In the end, we made the controversial and
difficult decision to abandon the Kinect in production this year due to
extensive cost-benefit analysis determining that the opportunity cost of
diverting precious build season time to it was too great of a risk.\@


\subsection{Driver Controls}

We also focused on creating an improved driver experience this year. One
of our main issues in previous years was the lack of driver practice and
experience operating the robot and its control interface. This was
partly due to poor time management on our part and partly due to
unintuitively designed controls and user interfaces. One of our main
goals this year was to create a slick and easy driver experience to
reduce as much human error and friction as possible. We implemented
multiple separate drive modes and selectors to allow a customizable
experience. The de facto standard for FRC robot driver dashboard is the
Shuffleboard, included as part of the FRC Game Tools software
distribution from LabView. This piece of archaic technology has an
outdated user interface that can only provide rudimentary data and
visualizations from the robot. This season however, we decided to create
our own proprietary driver dashboard for the first time. This dashboard,
dubbed ``Jankboard,'' has features like collision detection, 3D robot
visualization, and links directly to our other driver features like
speed and drive modes.

% TODO: add more sections

\section{Drivebase}

% TODO: add yapping about the drivebase and how good it is

\blindtext{}
\blindtext{}


\section{Vision}

% TODO: add yapping about vision

\subsection{How we build BozoVision}

% TODO: add deeper implementation details about vision

\blindtext{}
\blindtext{}

\section{Driver Controls}

One of our biggest weaknesses in previous years was with the driver's
mastery (or lack thereof) over our robot's controls. No matter how well
the robot was designed, it was ultimately held back by a lack of driver
familiarity with the control systems. We aimed to avoid this issue this
year by investing heavily into the driver's controls.

\subsection{Drive modes and the gear selector}

One of our major developments was the gear/drive mode selector system
(these terms may be used interchangeably throughout this paper). With
clear and distinct ``gears'' separating each of our operating modes,
mental stress on the driver is vastly reduced, as they no longer have to
think about whether they have some feature like ``reverse enabled,'' but
instead can simply think about what ``gear'' they are in. This subtle
but important paradigm shift allows the driver to easily choose the
right settings to fulfill their current needs without thinking about
convoluted control schemes.

\subsubsection{How we created drive modes}
% TODO: talk more about the drive modes and their implementation in
% code, and their streaming via network tables
\blindtext{}

\subsection{Dashboard}

Something we worked on this season which many FRC teams often overlook
was the driver dashboard. An amazing robot and control scheme mean
nothing if vital information is not quickly and intuitively conveyed to
the driver. With this in mind, we set out to design our own dashboard
which was tightly integrated with our specific robot controls and
features to bring out their fullest potential, dubbed the ``Jankboard''.
Despite being christened after the term ``jank,'' the Jankboard is
anything but. It has a full 3D driver visualization that automatically
syncs and updates with the real-time status of the robot, and responds
to user actions like changing drive or speed modes. It also warns the
user of imminent collision, and reports useful telemetry information
like speed, heading, etc.

\subsubsection{How we built Jankboard}

Jankboard is one of our only programming projects not made with Java.
Its frontend is build using the Svelte framework and TypeScript, and its
backend which handles calculations and telemetry uplink with our robot
is written in Rust, using a custom FRC NetworkTables v4 implementation.


% TODO: discuss the genius of jankboard


\printbibliography

\end{document}